\IfFileExists{Pacotes/longquadro.sty}{\usepackage{Pacotes/longquadro}}{}
\IfFileExists{Pacotes/longtab.sty}{\usepackage{Pacotes/longtab}\let\longtable\longtabela\relax\let\endlongtable\endlongtabela\relax}{}
\usepackage{titletoc}%formatar sumario
\usepackage{newfloat}%novos ambientes/listas
\usepackage{subfig}
\usepackage{listings}
%===========================================================================
%===========================================%ajuste em ambientes
\SetupFloatingEnvironment{figure}{
	placement = H,
	chapterlistsgaps=off}
\renewcommand{\thefigure}{\arabic{figure}}
%=====================================
\SetupFloatingEnvironment{table}{
	placement = H,
	chapterlistsgaps=off}
\renewcommand{\thetable}{\arabic{table}}
%===========================================================================
%===========================================%novos ambientes
%lista de quadros
\DeclareFloatingEnvironment[
	fileext = loq,
	listname = LISTA DE QUADROS,
	name = Quadro,
	placement = H,
	chapterlistsgaps=off
]{quadro}
\renewcommand{\thequadro}{\arabic{quadro}}
%=====================================
%lista de graficos
\DeclareFloatingEnvironment[
	fileext = ldg,
	listname = LISTA DE GRÁFICOS,
	name = Gráfico,
	placement = H,
	chapterlistsgaps=off
]{grafico}
\renewcommand{\thegrafico}{\arabic{grafico}}
%=====================================
%equações
\DeclareFloatingEnvironment[
	fileext = loe,
	listname = LISTA DE EQUAÇÕES,
	name = Equação,
	placement = H,
	chapterlistsgaps=off
]{equacao}
\renewcommand{\theequacao}{\arabic{equacao}}
\renewcommand{\theequation}{\arabic{equation}}
%=====================================
%lista de programas
\lstset{
	numberbychapter=false,
	backgroundcolor=\color{black}, 
	commentstyle=\color{green},
	keywordstyle=\color{magenta},
	numberstyle=\tiny\color{gray},
    	stringstyle=\color{purple}, 
	breaklines=true,
	basicstyle=\small\color{white}
}
\def\lstlistingname{Código}
\def\lstlistlistingname{Lista de códigos}
%===========================================================================
\luaexec{
	local listinha = {
		{"table",	"LTT",	"lot"},
		{"quadro",	"LQ",	"loq"}
	}
	for k, v in ipairs(listinha) do
		tex.sprint("\\makeatletter\\def\\"..v[2].."@c@ption\#1[\#2]\#3{ \\"	..v[2].."@makecaption\#1\\fnum@"..v[1].."{\#3}\\def\\@tempa{\#2}\\ifx\\@tempa\\@empty\\else {\\let\\\\\\space \\addcontentsline{"..v[3].."}{"..v[1].."}{\\protect\\numberline{\\the"..v[1].."}{\#2}}} \\fi} \\makeatother")
	end
}
%===========================================%formato
\luaexec{%
	local listinha = {
		{"chapter", 		"\\bfseries", 	"\\MakeUppercase"},
		{"section", 		"", 		"\\MakeUppercase"},
		{"subsection", 	"", 		""},
		{"subsubsection", 	"", 		""},
		{"paragraph", 	"", 		""}
	 }
	 for _, v in ipairs(listinha) do
	 	tex.sprint("\\titlecontents{"..v[1].."}[\\caixalargsec]{"..v[2].."}{\\contentslabel{\\caixalargsec}"..v[3].."}{}{\\space\\titlerule*[.3pc]{.}\\contentspage\\vspace{\\distanciasecs}}[]")
	 end
}

%===========================================
\luaexec{%
	local listinha = {
		"figure"	,
		"table"	,
		"quadro"	,
		"grafico"	,
		"equacao"	,
		"lstlisting"
	}
	for _, v in ipairs(listinha) do
		tex.sprint("\\titlecontents{"..v.."}[0cm]{}{\\"..v.."name\\space\\thecontentslabel\\space-\\space}{}{\\space\\titlerule*[.3pc]{.}\\contentspage\\vspace{\\distanciasecs}}[]")
	end
}
%===========================================================================
\newcommand{\printlistasdeilustracoes}{%
\listoffigures
%\newpage

\listoftables
%\newpage

\listofquadros
%\newpage

%\listofgraficos
%\newpage

%\listofequacaos
%\newpage

\lstlistoflistings
\newpage%
}

%(no) restart sempre que
\counterwithout{figure}{chapter}
\counterwithout{table}{chapter}
\counterwithout{quadro}{chapter}
\counterwithout{grafico}{chapter}
\counterwithout{equacao}{chapter}
\counterwithout{equation}{chapter}

%Configuração gráfica
\def\padraoilustra{%
\footnotesize%
\linespread{1}\selectfont%
\centering%

}

%floats
%\AddToHook{env/figure/begin}{\padraoilustra}
%\AddToHook{env/grafico/begin}{\footnotesize}
%\AddToHook{env/table/begin}{\footnotesize}
%\AddToHook{env/quadro/begin}{\padraoilustra}
%%\renewcommand{\longtablehook}{\padraoilustra}
%\renewcommand{\longquadrohook}{\padraoilustra}
%\UseHook{env/quadro/begin}

%==================================================================
\def\sumarioref{}
%apendice
\def\sumarioapendice{\titlecontents{chapter}[\caixalargsec]{\bfseries}{APÊNDICE\space\thecontentslabel\space-\space\MakeUppercase}%
{}{\space\titlerule*[.3pc]{.}\contentspage\vspace{\distanciasecs}}%
[\phantomsection]}
%anexo
\def\sumarioanexo{\titlecontents{chapter}[\caixalargsec]{\bfseries}{ANEXO\space\thecontentslabel\space-\space\MakeUppercase}%
{}{\space\titlerule*[.3pc]{.}\contentspage\vspace{\distanciasecs}}%
[\phantomsection]}

\typeout{sumario e listas atualizados}