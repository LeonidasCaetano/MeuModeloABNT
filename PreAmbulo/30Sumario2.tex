\IfFileExists{Pacotes/longquadro.sty}{\usepackage{Pacotes/longquadro}}{}
\IfFileExists{Pacotes/longtab.sty}{\usepackage{Pacotes/longtab}\let\longtable\longtabela\relax\let\endlongtable\endlongtabela\relax}{}
\usepackage{titletoc}%formatar sumario
\usepackage{newfloat}%novos ambientes/listas
\usepackage{listings}
%===========================================================================
%===========================================%ajuste em ambientes
\SetupFloatingEnvironment{figure}{
	placement = H,
	chapterlistsgaps=off}
\renewcommand{\thefigure}{\arabic{figure}}
%=====================================
\SetupFloatingEnvironment{table}{
	placement = H,
	chapterlistsgaps=off}
\renewcommand{\thetable}{\arabic{table}}

\renewcommand{\thetable}{\arabic{table}}
\makeatletter
\def\LTT@c@ption#1[#2]#3{%
  \LTT@makecaption#1\fnum@table{#3}%
  \def\@tempa{#2}%
  \ifx\@tempa\@empty\else
     {\let\\\space
     \addcontentsline{lot}{table}{\protect\numberline{\thetable}{#2}}}%
  \fi}
\makeatother
%===========================================================================
%===========================================%novos ambientes
%lista de quadros
\DeclareFloatingEnvironment[
	fileext = loq,
	listname = LISTA DE QUADROS,
	name = Quadro,
	placement = H,
	chapterlistsgaps=off
]{quadro}
\renewcommand{\thequadro}{\arabic{quadro}}
\makeatletter
\def\LQ@c@ption#1[#2]#3{%
  \LQ@makecaption#1\fnum@quadro{#3}%
  \def\@tempa{#2}%
  \ifx\@tempa\@empty\else
     {\let\\\space
     \addcontentsline{loq}{quadro}{\protect\numberline{\thequadro}{#2}}}%
  \fi}
\makeatother
%=====================================
%lista de graficos
\DeclareFloatingEnvironment[
	fileext = ldg,
	listname = LISTA DE GRÁFICOS,
	name = Gráfico,
	placement = H,
	chapterlistsgaps=off
]{grafico}
\renewcommand{\thegrafico}{\arabic{grafico}}
%=====================================
%lista de programas
\lstset{
	numberbychapter=false,
	backgroundcolor=\color{black}, 
	commentstyle=\color{green},
	keywordstyle=\color{magenta},
	numberstyle=\tiny\color{gray},
    	stringstyle=\color{purple}, 
	breaklines=true,
	basicstyle=\small\color{white}
}
\def\lstlistingname{Código}
\def\lstlistlistingname{Lista de códigos}
%===========================================================================
%===========================================%formato
%chapter
\titlecontents{chapter}[\caixalargsec]
{\bfseries}
{\contentslabel{\caixalargsec}\MakeUppercase}
{}
{\space\titlerule*[.3pc]{.}\contentspage\vspace{\distanciasecs}}
[]
%===========================================
%seção
\titlecontents{section}[\caixalargsec]
{}
{\contentslabel{\caixalargsec}\MakeUppercase}
{}
{\space\titlerule*[.3pc]{.}\contentspage\vspace{\distanciasecs}}
[]
%===========================================
%subseção
\titlecontents{subsection}[\caixalargsec]
{}
{\contentslabel{\caixalargsec}}
{}
{\space\titlerule*[.3pc]{.}\contentspage\vspace{\distanciasecs}}
[]
%===========================================
%4º
\titlecontents{subsubsection}[\caixalargsec]
{}
{\contentslabel{\caixalargsec}}
{}
{\space\titlerule*[.3pc]{.}\contentspage\vspace{\distanciasecs}}
[]
%===========================================
%5º
\titlecontents{paragraph}[\caixalargsec]
{}
{\contentslabel{\caixalargsec}}
{}
{\space\titlerule*[.3pc]{.}\contentspage\vspace{\distanciasecs}}
[]
%===========================================
%figuras
\titlecontents{figure}[0cm]
{}
{Figura\space\thecontentslabel\space-\space}
{}
{\space\titlerule*[.3pc]{.}\contentspage\vspace{\distanciasecs}}
[]
%===========================================
%tabelas
\titlecontents{table}[0cm]
{}
{Tabela\space\thecontentslabel\space-\space}
{}
{\space\titlerule*[.3pc]{.}\contentspage\vspace{\distanciasecs}}
[]
%===========================================
%quadros
\titlecontents{quadro}[0cm]
{}
{Quadro\space\thecontentslabel\space-\space}
{}
{\space\titlerule*[.3pc]{.}\contentspage\vspace{\distanciasecs}}
[]
%===========================================
%graficos
\titlecontents{grafico}[0cm]
{}
{Gráfico\space\thecontentslabel\space-\space}
{}
{\space\titlerule*[.3pc]{.}\contentspage\vspace{\distanciasecs}}
[]
%===========================================
%equações
\titlecontents{lstlisting}[0cm]
{}
{\lstlistingname\space\thecontentslabel\space-\space}
{}
{\space\titlerule*[.3pc]{.}\contentspage\vspace{\distanciasecs}}
[]
%===========================================
%===========================================================================
\newcommand{\printlistasdeilustracoes}{%
\listoffigures
%\newpage

\listoftables
%\newpage

\listofquadros
%\newpage

%\listofgraficos
%\newpage

\lstlistoflistings
\newpage%
}

%(no) restart sempre que
\counterwithout{figure}{chapter}
\counterwithout{table}{chapter}
\counterwithout{quadro}{chapter}
\counterwithout{grafico}{chapter}

%Configuração gráfica
\def\padraoilustra{%
\footnotesize%
\linespread{1}\selectfont%
\centering%

}

%floats
%padrão ilustra under dev

%==================================================================
%apendice
\def\sumarioapendice{\titlecontents{chapter}[\caixalargsec]{\bfseries}{APÊNDICE\space\thecontentslabel\space-\space\MakeUppercase}%
{}{\space\titlerule*[.3pc]{.}\contentspage\vspace{\distanciasecs}}%
[\phantomsection]}
%anexo
\def\sumarioanexo{\titlecontents{chapter}[\caixalargsec]{\bfseries}{ANEXO\space\thecontentslabel\space-\space\MakeUppercase}%
{}{\space\titlerule*[.3pc]{.}\contentspage\vspace{\distanciasecs}}%
[\phantomsection]}

\typeout{sumario e listas atualizados}