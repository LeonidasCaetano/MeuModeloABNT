%==================================================================
%pacotes
\usepackage[english,brazil]{babel}%Pacote BR
%=========================================================
\usepackage[normalem]{ulem}%sublinhar, etc
\usepackage{soul}
%=========================================================
\usepackage{mathtools}%equações matematicas
\usepackage{amsfonts}%letras matematicas
\usepackage{siunitx}\sisetup{detect-all}%Unidades
\usepackage{polynom}%divisão polinomial
\usepackage{chemfig}%quimica 
%=========================================================
\usepackage{multicol}%multicoluna
\usepackage{graphicx}%Inserir imagem
\usepackage{xcolor}%cores
%=========================================================
%Tabelas
\usepackage{multirow}%multilinha
\usepackage{array}%basicão
\usepackage{tabularx}%tabela estranha
\usepackage{colortbl}%cor na cll
\usepackage{booktabs}%solo for py
%=========================================================
\usepackage[T1]{fontenc}%acentos
\usepackage{setspace}%definir distancia entre linhas, etc
\usepackage{anyfontsize}
\usepackage{indentfirst}%recuo 1º paragrafo
\usepackage{parskip}%distancia entre paragrafos
%=========================================================
%\usepackage{qrcode}%QRCode
%=========================================================
\usepackage{calc}%operação com variaveis
\usepackage[section]{placeins}
\usepackage{float}%posição absoluta
\usepackage[absolute]{textpos}%posicionar textos livremente
%=========================================================
\usepackage{datatool}%organizar listas
\usepackage{etoolbox}
\usepackage{imakeidx}\makeindex[columns=1, title=ÍNDICE,intoc]
%=========================================================
\ifdefined\directlua
	\usepackage{fontspec}%mudar fonte
	\usepackage{pdflscape}
	\usepackage{luacode}
	\usepackage{Pacotes/luasiglas}
\fi
\ifdefined\pdfoutput
	\usepackage{helvet}
	\renewcommand{\familydefault}{\sfdefault}%Arial
	\usepackage{pdflscape}
\fi
\ifdefined\XeTeXversion
	\usepackage{mathspec}%mudar fonte
	\usepackage{lscape}%pag deitada
\fi
%\usepackage{contour}%fonte aberta
%=========================================================
\usepackage{lipsum}%txt randomico
\usepackage{pgffor}%loop option 1 
\usepackage{forloop}%loop option 2
%=========================================================
\usepackage{appendix}
%=========================================================
\usepackage[text=Fiado\space só \\dia\space  30/02]{draftwatermark}%marca d'gua
\usepackage[percent]{overpic}%escrever sobre fts
%==================================================================

\typeout{Pacotes carregados}