\usepackage{longtable}
\IfFileExists{longquadro.sty}{\usepackage{longquadro}}{\newenvironment{longquadro}{\begin{longtable}}{\end{longtable}}}
\usepackage{titletoc}%formatar sumario
\usepackage{newfloat}%novos ambientes/listas
\usepackage{listings}
%===========================================================================
%===========================================%ajuste em ambientes
\SetupFloatingEnvironment{figure}{
	fileext = lol,
	placement = H,
	chapterlistsgaps=off
}
\renewcommand{\thefigure}{\arabic{figure}}
\SetupFloatingEnvironment{table}{
	fileext = lol,
	placement = H,
	chapterlistsgaps=off
}
\renewcommand{\thetable}{\arabic{table}}
%===========================================================================
%===========================================%novos ambientes
%lista de quadros
\DeclareFloatingEnvironment[
    fileext = lol,
    listname = LISTA DE QUADROS,
    name = Quadro,
    placement = H,
    chapterlistsgaps=off
]{quadro}
\renewcommand{\thequadro}{\arabic{quadro}}
\makeatletter
\def\LQ@c@ption#1[#2]#3{%
  \LQ@makecaption#1\fnum@quadro{#3}%
  \def\@tempa{#2}%
  \ifx\@tempa\@empty\else
     {\let\\\space
     \addcontentsline{lol}{quadro}{\protect\numberline{\thequadro}{#2}}}%
  \fi}
\makeatother
%=====================================
%lista de graficos
\DeclareFloatingEnvironment[
    fileext = lol,
    listname = LISTA DE GRÁFICOS,
    name = Gráfico,
    placement = H,
    chapterlistsgaps=off
]{grafico}
\renewcommand{\thegrafico}{\arabic{grafico}}
%=====================================
%equações
\DeclareFloatingEnvironment[
    fileext = lol,
    listname = LISTA DE EQUAÇÕES,
    name = Equação,
    placement = H,
    chapterlistsgaps=off
]{equacao}
\renewcommand{\theequacao}{\arabic{equacao}}
\renewcommand{\theequation}{\arabic{equation}}
%=====================================
%lista de programas
\lstset{numberbychapter=false,
backgroundcolor=\color{black}, 
commentstyle=\color{green},
    keywordstyle=\color{magenta},
    numberstyle=\tiny\color{gray},
    stringstyle=\color{purple}, 
    breaklines=true,
    basicstyle=\small\color{white}
}
\def\lstlistingname{Código}
\def\lstlistlistingname{Lista de códigos}
%=====================================
\DeclareFloatingEnvironment[
    fileext = lol,
    listname = LISTA DE PROCESSOS,
    name = Processo,
    placement = H,
    chapterlistsgaps=off
]{processo}
\renewcommand{\theprocesso}{\arabic{processo}}
\DeclareFloatingEnvironment[
    fileext = lol,
    name = Subprocesso,
    placement = H,
    chapterlistsgaps=off
]{subprocesso}
\renewcommand{\thesubprocesso}{\arabic{processo}.\arabic{subprocesso}}
%===========================================================================
%===========================================%Lista principal
\DeclareFloatingEnvironment[
    fileext = lol,
    listname = LISTA DE ILUSTRAÇÕES,
    name = Ilustração,
    placement = H,
    chapterlistsgaps=off
]{ilustracao}
%===========================================================================
%===========================================%formato
%secão
\titlecontents{chapter}[\caixalargsec]
{\bfseries}
{\contentslabel{\caixalargsec}\MakeUppercase}
{}
{\space\titlerule*[.3pc]{.}\contentspage\vspace{\distanciasecs}}
[]
%===========================================
%subseção
\titlecontents{section}[\caixalargsec]
{}
{\contentslabel{\caixalargsec}\MakeUppercase}
{}
{\space\titlerule*[.3pc]{.}\contentspage\vspace{\distanciasecs}}
[]
%===========================================
\titlecontents{subsection}[\caixalargsec]
{}
{\contentslabel{\caixalargsec}}
{}
{\space\titlerule*[.3pc]{.}\contentspage\vspace{\distanciasecs}}
[]
%===========================================
\titlecontents{subsubsection}[\caixalargsec]
{}
{\contentslabel{\caixalargsec}}
{}
{\space\titlerule*[.3pc]{.}\contentspage\vspace{\distanciasecs}}
[]
%===========================================
\titlecontents{paragraph}[\caixalargsec]
{}
{\contentslabel{\caixalargsec}}
{}
{\space\titlerule*[.3pc]{.}\contentspage\vspace{\distanciasecs}}
[]
%===========================================
%figuras
\titlecontents{figure}[0cm]
{}
{Figura\space\thecontentslabel\space-\space}
{}
{\space\titlerule*[.3pc]{.}\contentspage\vspace{\distanciasecs}}
[]
%===========================================
%tabelas
\titlecontents{table}[0cm]
{}
{Tabela\space\thecontentslabel\space-\space}
{}
{\space\titlerule*[.3pc]{.}\contentspage\vspace{\distanciasecs}}
[]
%===========================================
%quadros
\titlecontents{quadro}[0cm]
{}
{Quadro\space\thecontentslabel\space-\space}
{}
{\space\titlerule*[.3pc]{.}\contentspage\vspace{\distanciasecs}}
[]
%===========================================
%graficos
\titlecontents{grafico}[0cm]
{}
{Gráfico\space\thecontentslabel\space-\space}
{}
{\space\titlerule*[.3pc]{.}\contentspage\vspace{\distanciasecs}}
[]
%===========================================
%equações
\titlecontents{equacao}[0cm]
{}
{Equação\space\thecontentslabel\space-\space}
{}
{\space\titlerule*[.3pc]{.}\contentspage\vspace{\distanciasecs}}
[]
%===========================================
%equações
\titlecontents{lstlisting}[0cm]
{}
{\lstlistingname\space\thecontentslabel\space-\space}
{}
{\space\titlerule*[.3pc]{.}\contentspage\vspace{\distanciasecs}}
[]

%equações
\titlecontents{processo}[0cm]
{}
{Processo\space\thecontentslabel\space-\space}
{}
{\space\titlerule*[.3pc]{.}\contentspage\vspace{\distanciasecs}}
[]
%===========================================
%equações
\titlecontents{subprocesso}[0cm]
{}
{Subprocesso\space\thecontentslabel\space-\space}
{}
{\space\titlerule*[.3pc]{.}\contentspage\vspace{\distanciasecs}}
[]
%===========================================================================
\newcommand{\printlistasdeilustracoes}{%
\listofilustracaos%
}

%(no) restart sempre que
\counterwithout{figure}{chapter}
\counterwithout{table}{chapter}
\counterwithout{quadro}{chapter}
\counterwithout{grafico}{chapter}
\counterwithout{equacao}{chapter}
\counterwithout{equation}{chapter}

%Configuração gráfica
\def\padraoilustra{%
\footnotesize%
\linespread{1}\selectfont%
\centering%
}

%floats

\atbeginenvironment{figure}{\padraoilustra}
\atbeginenvironment{table}{\padraoilustra}
\atbeginenvironment{quadro}{\padraoilustra}
\atbeginenvironment{longtable}{\padraoilustra}
\atbeginenvironment{longquadro}{\padraoilustra}
\atbeginenvironment{grafico}{\padraoilustra}

%==================================================================
\def\sumarioref{}
%apendice
\def\sumarioapendice{\titlecontents{chapter}[\caixalargsec]{\bfseries}{APÊNDICE\space\thecontentslabel\space-\space\MakeUppercase}%
{}{\space\titlerule*[.3pc]{.}\contentspage\vspace{\distanciasecs}}%
[\phantomsection]}
%anexo
\def\sumarioanexo{\titlecontents{chapter}[\caixalargsec]{\bfseries}{ANEXO\space\thecontentslabel\space-\space\MakeUppercase}%
{}{\space\titlerule*[.3pc]{.}\contentspage\vspace{\distanciasecs}}%
[\phantomsection]}

\typeout{sumario e listas atualizados}