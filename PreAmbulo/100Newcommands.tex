%==================================================================
%novos comandos
%==================================================================
%definemodel
\newcommand{\tit}{\hspace*{1cm}}
\newcommand{\subtit}{}
\newcommand{\loc}{\hspace*{1cm}}
\newcommand{\inst}{\hspace*{1cm}}
\newcommand{\campi}{\hspace*{1cm}}
\newcommand{\autum}{\hspace*{1cm}}
\newcommand{\autdos}{\hspace*{1cm}}
\newcommand{\auttre}{\hspace*{1cm}}
\newcommand{\autqua}{\hspace*{1cm}}
\newcommand{\ori}{\hspace*{1cm}}
\newcommand{\coori}{\hspace*{1cm}}
\newcommand{\nota}{\hspace*{1cm}}
\newcommand{\palavraschave}{}
%========================
\newcommand{\hyperreftit}{Modelo ABNT}
\newcommand{\hyperrefsubtit}{}
\newcommand{\hyperrefautum}{Leonidas Caetano}
\newcommand{\hyperrefautdos}{}
\newcommand{\hyperrefauttre}{}
\newcommand{\hyperrefautqua}{}
\newcommand{\hyperrefautores}{{\hyperrefautum; \hyperrefautdos;\hyperrefauttre; \hyperrefautqua}}
\newcommand{\hyperrefkeywords}{ABNT; LaTeX}
%========================
\newcommand{\epigraphy}{\parbox{10cm}{``I heard a definition of an intellectual, that I thought was very interesting: a man who takes more words than are necessary to tell more than he knows."\par Dwight D. Eisenhower}}
\newcommand{\dedique}{A família}
%========================
\newcommand{\bancaum}{medo}
\newcommand{\bancados}{me salva}
%================================================
\newcommand{\titulo}[1]{\renewcommand{\hyperreftit}{#1}\renewcommand{\tit}{#1}}
\newcommand{\subtitulo}[1]{\renewcommand{\hyperrefsubtit}{#1}\renewcommand{\subtit}{\ifstrequal{#1}{}{}{: {#1}}}}
\newcommand{\local}[1]{\renewcommand{\loc}{#1}}
\newcommand{\institui}[1]{\renewcommand{\inst}{\ifstrequal{#1}{}{\hspace*{1cm}}{\MakeUppercase{#1}}}}
\newcommand{\campus}[1]{\renewcommand{\campi}{\ifstrequal{#1}{}{\hspace*{1cm}}{\MakeUppercase{#1}}}}
\newcommand{\autorum}[1]{\renewcommand{\hyperrefautum}{#1}\renewcommand{\autum}{\ifstrequal{#1}{}{\hspace*{1cm}}{#1}}}
\newcommand{\autordos}[1]{\renewcommand{\hyperrefautdos}{#1}\renewcommand{\autdos}{\ifstrequal{#1}{}{\hspace*{1cm}}{#1}}}
\newcommand{\autortre}[1]{\renewcommand{\hyperrefauttre}{#1}\renewcommand{\auttre}{\ifstrequal{#1}{}{\hspace*{1cm}}{#1}}}
\newcommand{\autorqua}[1]{\renewcommand{\hyperrefautqua}{#1}\renewcommand{\autqua}{\ifstrequal{#1}{}{\hspace*{1cm}}{#1}}}
\newcommand{\orientador}[1]{\renewcommand{\ori}{\ifstrequal{#1}{}{\hspace*{1cm}}{#1}}}
\newcommand{\coorientador}[1]{\renewcommand{\coori}{\ifstrequal{#1}{}{\hspace*{1cm}}{#1}}}
\newcommand{\notasandt}[1]{\renewcommand{\nota}{\ifstrequal{#1}{}{\hspace*{1cm}}{#1}}}
%========================
\newcommand{\palavraschaves}[1]{\renewcommand{\hyperrefkeywords}{{#1}}\renewcommand{\palavraschave}{{#1}}}
%========================
\newcommand{\banca}[2]{\renewcommand{\bancaum}{#1}\renewcommand{\bancados}{#2}}
%==================================================================
%datas
\newcommand{\data}{%
%\ifnum\the\day<10 0\fi\the\day/%
%\ifnum\the\month<10 0\fi\the\month/%
\the\year%
}
%==================================================================
%Fonte
\newcommand{\conteudofonte}{Os autores (\the\year)}
\newcommand{\posfonte}{\centering}

\newcommand{\fonte}[1][\conteudofonte]{

\posfonte
Fonte: #1%
}

%==================================================================
%seção não numerada
\newcommand{\secaon}[1]{%
\newpage

\phantomsection\label{#1}%
\addcontentsline{toc}{chapter}{\MakeUppercase{#1}}%
\chapter*{#1}%
}
%==================================================================
%seção pre listas
\newcommand{\secaop}[1]{%
\newpage

\phantomsection\label{#1}%
\chapter*{#1}%
}
%==================================================================
%titulo apendice
\newcounter{apendnumb}
\newcommand{\apend}{\secapendice\sumarioapendice}
%==================================================================
%anexo
\newcounter{anexnumb}
\newcommand{\anex}{\secanexo\sumarioanexo}
%==================================================================
%Tabelas py
\makeatletter
\@ifpackageloaded{booktabs}
{\typeout{Booktabs já foi carregado}}
{%
\typeout{Fazendo comandos de tabelas}
\newcommand{\toprule}{\noalign{\hrule height 2.2pt}}
\newcommand{\midrule}{\hline}
\newcommand{\cmidrule}[1]{\cline{#1}}
\newcommand{\bottomrule}{\noalign{\hrule height 2.2pt}}
}
\makeatother
%==================================================================
%referencias
\newcommand{\makeref}[1]{%
\begingroup
\linespread{1}\selectfont
\secaon{#1}

\printbibliography[title=#1,heading=none]%
\endgroup
}
%==================================================================
%Deitada
%=============================================
\ifdefined\pdfoutput
\newenvironment{deitada}{%
\newpage%
\begin{landscape}%
}{%
\end{landscape}%
\newpage%
}
\fi
%=============================================
\ifdefined\directlua
\newenvironment{deitada}{%
\newpage%
\begin{landscape}%
}{
\end{landscape}%
\newpage%
}
\fi
%=============================================
\ifdefined\XeTeXversion
\newenvironment{deitada}{%
\newpage%
%\eject \pdfpagewidth=29.7cm \pdfpageheight=21cm%
\newgeometry{
includehead = false,
includefoot = false,
twoside = false,
footskip = 1cm,
headheight = 2cm,
headsep = .5cm,
top = 3cm,
bottom = 10cm,
left = 3cm,
right = -6.7cm}%
\fancyhf{}%
\fancyhead[L]{}%
\fancyhead[C]{}%
\fancyhead[R]{\special{papersize=29.7cm,21cm}\footnotesize\thepage\hspace{-7.7cm}}%
\fancyfoot[L]{}%
\fancyfoot[C]{}%
\fancyfoot[R]{}%
}{%
\newpage%
%\eject \pdfpagewidth=21cm \pdfpageheight=29.7cm%
%==================================================================
%cabeçalho e rodapé
\pagestyle{fancy}
\fancyhf{}

\fancyhead[L]{}
\fancyhead[C]{}
\fancyhead[R]{\footnotesize\hyperref[sumario]{\thepage}\hspace{1cm}}
\fancyfoot[L]{}
\fancyfoot[C]{}
\fancyfoot[R]{}

%===========================================
\renewcommand{\headrule}{}%
\newgeometry{
twoside = false,
includehead = false,
includefoot = false,
twoside = false,
footskip = 1cm,
headheight = 3cm,
headsep = .5cm,
top = 3cm,
bottom = 2cm,
left = 3cm,
right = 2cm
}%
}
\fi
%==================================================================
%Formatar usando Lua
\ifdefined\directlua
	\directlua{require("PreAmbulo/101LuaCmds.lua")}
	
	\newcommand{\formatarporcent}[1]{\luaexec{
			tex.sprint(formatar_n_casas(100*#1,2))}\%}

	\newcommand{\formatarinteiro}[1]{\luaexec{
			tex.sprint(formatar_n_casas(#1,0))}}

	\newcommand{\formatarduascasas}[1]{\luaexec{
			tex.sprint(formatar_n_casas(#1,2))}}

	\newcommand{\formatarmoeda}[1]{R\$ \luaexec{
			tex.sprint(formatar_n_casas(#1,2))}}

	\newcommand{\formataroutro}[2]{\luaexec{
			tex.sprint(formatar_n_casas(#1,#2))}}

	\newcommand{\formatartexto}[1]{\luaexec{
			tex.sprint(#1)}}
\fi
%==================================================================

\typeout{precomandos feitos}